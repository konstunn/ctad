
\documentclass[a4paper,12pt]{article}

\usepackage{cmap}

\usepackage[T2A]{fontenc}
\usepackage[utf8x]{inputenc}
\usepackage[russian]{babel}

\usepackage[a4paper,margin=1.5cm,footskip=1cm,left=2cm,right=1.5cm,top=1.5cm
	,bottom=2.0cm]{geometry}
\usepackage{textcase}
\usepackage{csquotes}
\usepackage{enumitem}

\usepackage{ucs}                                              %%
\usepackage{color}                                            %%
\usepackage{array}                                            %%
\usepackage{longtable}                                        %%
\usepackage{calc}                                             %%
\usepackage{multirow}                                         %%
\usepackage{hhline}                                           %%
\usepackage{ifthen}                                           %%
%%  optionally (for landscape tables embedded in another document): %%
\usepackage{lscape}                                           %%

\usepackage{caption}

\usepackage{amsmath}
\usepackage{pgfplots}

\usepackage{float}

%\usepackage{extsizes}

\setlist[description]{leftmargin=\parindent,labelindent=\parindent}

\begin{document}

\setcounter{secnumdepth}{0}

\begin{titlepage}

	\begin{center}
		Новосибирский государственный технический университет
		
		Факультет прикладной математики и информатики
		
		Кафедра теоретической и прикладной информатики
		
		\vspace{250pt}
		
		\textbf{\LARGE{Лабораторная работа № 3}}
		\medbreak
		\large{<<Экспериментальное исследование исследования свойств критерия 
			$\chi^2$ Пирсона>>}\\
		\medbreak
		по дисциплине\\
		\medbreak
		<<Компьютерные технологии анализа данных и исследования 
			статистических~закономерностей>>
		\vspace{150pt}
	\end{center}

	\begin{flushleft}
		\begin{tabbing}
			Группа:\qquad\qquad \= ПММ-61\\
			Студент:            \> Горбунов К. К.\\
			Преподаватель:      \> проф. Лемешко Б. Ю.\\
			Вариант:            \> 12 \\
		\end{tabbing}
	\end{flushleft}

	\begin{center}
		\vspace{\fill}
		Новосибирск, 2016 г.
	\end{center}

\end{titlepage}

\newpage

\section{Цель работы}

Исследование влияния способов группирования на предельные распределения
статистики критерия согласия $\chi^2$ Пирсона при проверки простых и сложных
гипотез (при использовании для вычисления оценок по негруппированным данным
метода максимального правдоподобия.

\section{Краткие теоретические сведения}

Сведения.

\section{Ход работы}

Вариант 12

\begin{enumerate}

\item ход работы
\item ход работы

\end{enumerate}

\section*{Заключение}

Заключение.

\section*{Список источников}
\addcontentsline{toc}{section}{Список источников}

1. Статистический анализ данных, моделирование и исследование вероятностных
закономерностей. Компьютерный подход : монография / Б.Ю. Лемешко, С.Б. Лемешко,
\mbox{С.Н. Постовалов}, Е.В Чимитова. --- Новосибирск : Изд-во НГТУ, 2011. --- 888 с.
(серия <<\mbox{Монографии НГТУ}>>).

\end{document}

# vim: ts=2 sw=2
